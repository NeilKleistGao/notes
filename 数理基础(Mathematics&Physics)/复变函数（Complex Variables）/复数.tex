\documentclass[UTF8]{article}
\usepackage{ctex}

\begin{document}
    \section{代数性质}
        \subsection{和与积}
            任意一个复数$z=(x, y)$可以写为$z = (x, 0) + (0, 1)(y, 0)$。

            实数的加法单位元$0 = (0, 0)$和乘法单位元$1 = (1, 0)$同样适用于
            复数。

            对于每一个复数$z = (x, y)$,存在一个加法逆元$-z = (-x, -y)$。
            对于任意的非零复数$z = (x, y)$,存在一个复数$z^{-1}$,使得$zz^{-1} = 1$。

            $z^{-1} = (\frac{x}{x^2 + y^2}, \frac{-y}{x^2 + y^2})$。
        \subsection{其他代数性质}
            如果$z_1z_2 = 0$,两个复数中至少有一个是0。

            $z_1^{-1}z_2^{-1} = (z_1z_2)^{-1} = \frac{1}{z_1z_2}$。

            涉及实数的二项式公式对复数仍然成立,即:
            $$(z_1 + z_2)^n = \sum_{k = 0}^{n}C_n^kz_1^kz_2^{n - k}$$
    \section{复平面与向量}
        \subsection{向量和模}
            $$Rez \leq |Rez| \leq |z|, Imz \leq |Imz| \leq |z|$$
        \subsection{三角不等式}
            $$|z_1 + z_2| \leq |z_1| + |z_2|, |z_1 + z_2| \geq ||z_1| - |z_2||$$

            证明:$|z^n| = |z|^n, (n = 1, 2, \dots)$(数学归纳法)。
        \subsection{共轭}
            一个复数$z = x + iy$的共轭复数$\bar z = x -iy$,是关于实轴的对称点。
        \subsection{复平面中的区域}
            邻域:$|z - z_0| < \epsilon$,去心邻域:$0 < |z - z_0| < \epsilon$。
            如果存在$z_0$的一个邻域内只包含集合S的点,那么称$z_0$是集合S的一个内点。
            如果存在$z_0$的一个邻域内不包含集合S的点,则称$z_0$是集合S的一个外点。
            如果$z_0$既不是内点也不是外点,则$z_0$为集合S的边界点,所有边界点的总和称为S的边界。

            如果一个集合不包含他的边界点,则这个集合是开集。一个集合是开集,当且仅当他的每个点都是他的内点。
            如果一个集合包含他的边界点,则称该集合是闭集。
            S集合的闭包是一个闭集。

            如果开集S的任意两个点$z_1$和$z_2$都可以由一条折线连接,并且这条折线由包含在S内的有限条
            线段首尾顺次相连组成,则称S是连通的。连通的非空开集称为开域。任何一个邻域都是开域。

            如果$z_0$的每一个去心邻域都至少包含S的一个点,那么称$z_0$是集合S的一个聚点或极限点。
    \section{指数形式}
        \subsection{指数形式}
            设$r$和$\theta$是非零复数$z = x + iy$所对应点的极坐标,$z = r(cos\theta + isin\theta)$。

            $\theta$的每一个值都称为$z$的辐角,这些值的集合用$argz$来表示,其主值记作$Argz$,是唯一满足$-\pi \leq \Theta \leq \pi$的值。

            欧拉公式:
            $$e^{i\theta} = cos\theta + isin\theta$$

            棣莫弗公式:
            $$(cos\theta + i\sin\theta)^n = cosn\theta + isinn\theta$$

            将复数写成矩阵形式,该公式等价于一个二维旋转矩阵,每次旋转角度$\theta$,一共旋转$n$次。
        \subsection{乘积与商的辐角}
            恒等式:$arg(z_1z_2) = arg(z_1) + arg(z_2)$。

            拉格朗日三角恒等式:
            $$1 + cos\theta + cos2\theta + \dots + cosn\theta = \frac{1}{2} + \frac{sin[(2n + 1)\frac{\theta}{2}]}{2sin(\frac{1}{2}\theta)}$$
        \subsection{复数的根}
            两个非零复数相等,当且仅当:$r_1 = r_2$且$\theta_1 = \theta_2 + 2k\pi$。

            一个复数的$n$次根的辐角$\theta = \frac{\theta_0 + 2k\pi}{n}$。

\end{document}